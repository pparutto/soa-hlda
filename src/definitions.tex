% *** Théorie
% *** Exemples d'applications

\section{Definitions}
\label{sec:definitions}

The HLDA is an extension of the multiclass LDA.
It's goal is to allow heteroscediscity: The classes respective Covariance matrix
can be different.
First, we will explain the bases of the LDA.
Then we will analyze how the HLDA is an extension of these bases.

\subsection{Theory}

The main goal of the multiclass LDA is to find a predictor that will allow to find the class that best correspond to a sample X.
In order to do that, we need to find a separator and each class will be projected on it.
The main goal is to find the best separator, so for a sample X, it will be easy to find which class
best correspond to X.
In a second time, it is possible to perform a dimension reduction on the separator.

To summarize, the LDA has two objectives:
- Find a separator that will allow a sample to be classified.
- Perform a dimension reduction to classify faster.

Now we will explain how the multiclass LDA works. 

\subsubsection{Multiclass LDA}

We suppose that we have a set of sample in $n$ dimension, labelized by their respective
class.
We suppose that the covariance matrix $\Sigma$ of all class is the same (homoscedasticity)
The mean of each class is notated $\nu$.
The number of different classes is $C$.

We want to find the class that best correspond to a vector x.
In order to easily differenciate classes, we need:

\begin{enumerate}
\item Each projected sample of the same class should be very close: we will call this the within-class ratio $S_W$.
\item The distance between each class data should be very large: we will call this the between-class ratio $S_B$.
\end{enumerate}

We will try to find the smallest within-class ratio and the biggest between-class ratio in order to best isolate each class.

So the function that we want to maximize is:
$$S = \frac{w^t S_B w}{w^t S_W w}$$
with:
\begin{itemize*}
  \item $S$ the ratio to maximize
  \item $S_B$ the between-class ratio
  \item $S_W$ the within-class ratio
  \item $w$ the projection matrix
\end{itemize*}

The expressions of $S_B$ and $S_W$ for the multiclass LDA are:
$$S_B = \frac{1}{C} * \sum\limits_{i = 1}^C (\mu_i - \mu)(\mu_i - \mu)^t$$
$$S_W = \frac{1}{C} * \sum\limits_{i = 1}^C \sum\limits_{\forall x \in C_i} (x - \mu_i)(x - \mu_i)^t$$
With:
\begin{itemize*}
  \item $\mu$ the mean of the class means
  \item $\mu_i$ the mean of the ith class
\end{itemize*}

For a multicall LDA, the best projection matrix for the ith dimension satisfy the equation:
$$(S_B - \lambda_i S_W)w_{i}^* = 0$$

This means that the projections with the maximum class separability information are the eigenvectors corresponding to the largest eigenvalues of $S_{W}^{-1} S_B$.

\subsubsection{HLDA}

HLDA (Heteroscedastic Linear Discriminant Analysis) is another method to find a linear transformation mapping the problem space into a smaller one.

As in the previous section, let $x$ be a sample of dimension $n$ and consider $C$ classes. Then the HLDA find the projection matrix A such that:
$$y = Ax$$
with y a $p$ dimensions variable and $p < n$.\\

The purpose of the HLDA method is to release the homoscedasticiy constraint of the LDA, allowing different variances between classes. The two assumptions made by the method are:

\begin{enumerate}
  \item The class distributions are $n$ dimensional Gaussian distributions.
  \item This space of size $n$ can be splitted into two subspaces of size $p$
    and $n - p$ (with p < n). The first one is different for all classes and contains meaningful information while the second one is the same for the whole dataset and contains only noise.
\end{enumerate}

The projection matrix A can then be rewrited following those assumptions:
$$A = \left [
  \begin{array}{c}
    A_p\\
    A_{n-p}
  \end{array}
\right ]
$$

it is also possible to find an expression for the mean and variance of each classes:
$$\mu_j =
\left [
  \begin{array}{c}
    \mu_{j,p}\\
    \mu_{0,(n-p)}
  \end{array}
\right ]
$$

$$\Sigma_j =
\left [
  \begin{array}{cc}
    \Sigma_j^p & 0\\
    0 & \Sigma_0^{(n-p)}\\
  \end{array}
\right ]
$$

Finally the Gaussian probability density function for this transformation is:
$$P(y_i) = \frac{\mid \theta \mid}{\sqrt{(2\pi)^n \mid \Sigma_{g(i)} \mid}}
    e^{-\frac{1}{2} (y_i - \mu_{g(i)})^t \Sigma_{g(i)}^{-1} (y_i - \mu_{g(i)})}$$
With $g(i)$ the function associating each sample with its label.

No straightforward analytical solution can be found to this equation as $\mu_j$ and $\Sigma_j$ may depend on A. In order to find A, the log-maximum likelihood method is used:
%$$L(\{x_i\}; A; \{\mu_j, \Sigma_j\}) = \sum\limits_{j=1}^{C} \sum_{i=1}{N_j}
