
\section{Variants of the HLDA}
\label{sec:variants}

There is several variants of the HLDA. We present them briefly.

First we will present a method which is a combination of the LDA and
the HDLA. In a second time, we present a method which allows to make our
computation faster.

\subsection{Smooth HLDA}
\label{sec:smooth-hlda}

If the HLDA works on a lot of classes, the estimation of the class
covariance becomes noisy. The LDA doesn't have this problem because
the within-class covariance matrix is the same for all the classes.

So, what is the solution when there is a lot of classes and the data
are not homoscedastics ? Burget\cite{burget.2004} presents a smooth
LDA. It is a combination of the LDA and the HLDA and it tries to get
the advantages of the two methods.

In this, we estimate the class covariance matrix as:

$$\breve\Sigma^{\left( j\right) }=\propto \hat\Sigma^{\left( j\right) } +\left( 1- \propto\right) \Sigma_{wc}$$

Where $\breve\Sigma^{\left( j\right) }$ is the smoothed estimated
covariance for the class $j$ used by the HLDA.
$\hat\Sigma^{\left(j\right) }$ is the covariance computed by the HLDA, and
$\Sigma_{wc}$ is the one computed by the LDA.

So it is clear that if $\propto$ is 1, the SHLDA is a HLDA, and if it
is equal to 1 it is a LDA.

%%% Local Variables:
%%% mode: latex
%%% TeX-master: "../hlda"
%%% End:
